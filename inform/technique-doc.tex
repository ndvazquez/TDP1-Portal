\documentclass[a4paper]{article}

\usepackage[utf8]{inputenc}
\usepackage[english]{babel}
\usepackage{fancyhdr,graphicx}
\usepackage[colorlinks=true,linkcolor=black,urlcolor=blue,bookmarksopen=true]{hyperref}

\newcommand{\materia}{[75.41] Taller de Programación I}
\newcommand{\trabajo}{Documentación técnica}
\newcommand{\trabajoheader}{Ejercicio final }
\newcommand{\cuatri}{1c2019}
\newcommand{\cuatrimestre}{Primer cuatrimestre de 2019}
\newcommand{\grupo}{Grupo 3}

\newcommand{\autores}{
	Camila Bojman,
	Cecilia Hortas,
	Nicolas Vazquez}

\hypersetup{
	pdftitle={\trabajo},
	pdfsubject={\materia},
	pdfauthor={\autores},
}

\pagestyle{fancy}
\fancyhf{}
\fancyhead[L]{\materia}
\fancyhead[R]{\trabajoheader - \trabajo}
\renewcommand{\headrulewidth}{0.4pt}
\fancyfoot[C]{\thepage}
\renewcommand{\footrulewidth}{0.4pt}

\begin{document}
	\pagenumbering{gobble}
	\pagenumbering{roman}
	\pagenumbering{arabic}
	\setcounter{page}{1}
	
	\begin{titlepage}
		\hfill\includegraphics[width=6cm]{fiuba.jpeg}
		\begin{center}
			\vfill
			\Huge \textbf{\trabajo}
			\vskip2cm
			\Large \materia\\
			\cuatrimestre
			\vfill
			\grupo
			\begin{itemize}
				\item Camila Bojman 101055 camiboj@gmail.com
				\item Cecilia Hortas 100687 ceci.hortas@gmail.com 100338 
				\item Nicolas Vazquez  vazquez.nicolas.daniel@gmail.com
			\end{itemize}
			\vskip1cm
		\end{center}
	\end{titlepage}

\section{Requerimientos de software}

En primer lugar el programa fue desarrollado enteramente en C++ en un sistema operativo Linux por lo que los comandos que se detallarán a continuación son para ese sistema operativo y sus distribuciones afines. Las bibliotecas utilizadas se presentan a continuación:

\begin{itemize}
	\item \texttt{Box 2D}: se encuentra en el repositorio por lo que basta con clonar el mismo
\end{itemize}

\begin{itemize}
	\item \texttt{Native JSON Benchmark}: igualmente se encuentra en el repositorio
\end{itemize}

\begin{itemize}
	\item \texttt{SDL}: se debe instalar a partir de los siguientes comandos:
\end{itemize}

\begin{verbatim}
sudo apt-get install libsdl2-dev
sudo apt-get install libsdl2-image-dev
\end{verbatim}

\begin{itemize}
	\item \texttt{SDL-mixer}: se debe instalar a partir del siguiente comando:
\end{itemize}

\begin{verbatim}
sudo apt-get install libsdl2-mixer-dev
\end{verbatim}

\begin{itemize}
	\item \texttt{YAML}: se debe instalar a partir del siguiente comando:
\end{itemize}

\begin{verbatim}
sudo apt-get install libyaml-cpp-dev
\end{verbatim}

Agregar algo relativo al sh para la compilación

\section{Descripción general}

El proyecto se constituye de los siguientes módulos:

\section{Servidor}

\subsection{Descripción general}

\subsection{Clases}

\subsection{Diagramas UML}

\subsection{Descripción de archivos y protocolos}

\section{Cliente}

\subsection{Descripción general}

\subsection{Clases}

\subsection{Diagramas UML}

\subsection{Descripción de archivos y protocolos}

\section{Editor}

\subsection{Descripción general}

\subsection{Clases}

\subsection{Diagramas UML}

\subsection{Descripción de archivos y protocolos}

\section{Programas intermedios y de prueba}

\section{Código fuente}


\end{document}