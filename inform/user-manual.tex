\documentclass[a4paper]{article}

\usepackage[utf8]{inputenc}
\usepackage[english]{babel}
\usepackage{fancyhdr,graphicx}
\usepackage[colorlinks=true,linkcolor=black,urlcolor=blue,bookmarksopen=true]{hyperref}

\newcommand{\materia}{[75.41] Taller de Programación I}
\newcommand{\trabajo}{Manual de usuario}
\newcommand{\trabajoheader}{Ejercicio final }
\newcommand{\cuatri}{1c2019}
\newcommand{\cuatrimestre}{Primer cuatrimestre de 2019}
\newcommand{\grupo}{Grupo 3}

\newcommand{\autores}{
	Camila Bojman,
	Cecilia Hortas,
	Nicolas Vazquez}

\hypersetup{
	pdftitle={\trabajo},
	pdfsubject={\materia},
	pdfauthor={\autores},
}

\pagestyle{fancy}
\fancyhf{}
\fancyhead[L]{\materia}
\fancyhead[R]{\trabajoheader - \trabajo}
\renewcommand{\headrulewidth}{0.4pt}
\fancyfoot[C]{\thepage}
\renewcommand{\footrulewidth}{0.4pt}

\begin{document}
	\pagenumbering{gobble}
	\pagenumbering{roman}
	\pagenumbering{arabic}
	\setcounter{page}{1}
	
	\begin{titlepage}
		\hfill\includegraphics[width=6cm]{fiuba.jpeg}
		\begin{center}
			\vfill
			\Huge \textbf{\trabajo}
			\vskip2cm
			\Large \materia\\
			\cuatrimestre
			\vfill
			\grupo
			\begin{itemize}
				\item Camila Bojman 101055 camiboj@gmail.com
				\item Cecilia Hortas 100687 ceci.hortas@gmail.
				\item Nicolas Vazquez 100338 vazquez.nicolas.daniel@gmail.com
			\end{itemize}
			\vskip1cm
		\end{center}
	\end{titlepage}

\section{Instalación}

\subsection{Requerimientos de software}

\subsection{Requerimientos de hardware}

\subsection{Proceso de instalación}

\section{Configuración}

\section{Forma de uso}
El juego es de tipo \textit{cooperativo}, lo que implica que para finalizar la partida todos deben alcanzar el objetivo reconocido como victoria. En este caso particular, \textbf{todos} los jugadores deben llegar a la llamada \textit{cake} que se trata de una torta localizada en algún lugar del escenario. Para ello deberán sortear con distintos obstáculos, en los cuales podrían morir. En el caso en que un jugador muera, esto no genera ninguna diferencia en la condición de victoria, incluso a veces puede ser necesario por ejemplo si todo el grupo llegó a la torta y un solo jugador quedó varado en el camino. En esta implementación del juego portal hay 3 causas de muerte:
\begin{enumerate}
	\item Si una bola de energia choca a Chell desde cualquier posición
	\item Si una roca cae encima de Chell
	\item Si Chell toca ácido accidentalmente
\end{enumerate}

A continuación se enumeran los distintos movimientos habilitados por las teclas del teclado:
\begin{itemize}
	\item \textbf{a} para moverse a la izquierda.
	\item \textbf{d} para moverse a la derecha.
	\item \textbf{w} para saltar.
	\item \textbf{g} para agarrar una roca. Un dato a tener en cuenta es que en caso de existir más de una roca en el escenario, se designará la que se encuentre a menor distancia del jugador en cuestión.
	\item \textbf{f} para dejar la roca en algún lugar en particular. Esto puede ser útil para abrir una compuerta (colocar una roca sobre un botón) o asesinar a otro jugador (tirándole una roca encima).
	\item \textbf{t} para lanzar una roca hacia arriba. Esta funcionalidad fue agregada para poder transportar una roca a través de un portal.
	\item \textbf{clic derecho} para lanzar un disparo con intenciones de colocar un portal de color naranja. 
	\item \textbf{clic izquierdo} para lanzar un disparo con intenciones de colocar un portal de color azul. 
	\item \textbf{r} para borrar los portales creados por el mismo jugador.
	\item \textbf{p} para colocar un \textit{pintool} en cualquier lugar de la pantalla. Para eso debe moverse el mouse hasta la posición determinada y luego presionar la tecla susodicha.
	\item \textbf{o} para eliminar el pintool puesto por el mismo jugador.
\end{itemize}

Ahora para poder localizar la torta es necesario moverse por el escenario, en el cual el usuario podría encontrarse con alguna dificultad u objeto desconocido. Se enumeran una serie de preceptos a tener en cuenta a la hora de jugar el juego:
\begin{itemize}
	\item Tener en cuenta que se puede crear un portal de color azul y un portal de color naranja por jugador y que sólo pueden crearse sobre los bloques de metal. Así mismo, la creación de un portal naranja elimina cualquier otro portal naranja creado \textbf{por el mismo jugador}. Obviamente no interfiere con portales creados por otros jugadores. Si el disparo no choca con algún bloque de metal, simplemente no hace nada.
	
	\item El ácido es letal. Como fue mencionado previamente, Chell al tocarlo se muere inmediatamente.
	
	\item Existen emisores y receptores de bolas de energía. El emisor lanza bolas de energía cada cierto período de tiempo, las cuales asesinan a Chell en el acto si se topan con ella. Hay 4 tipos de emisores: el que lanza bolas de energía para la izquierda, derecha, arriba o abajo. La idea de estas bolas de energía, además de representar un peligro para Chell, es llegar a un receptor de energía y activarlo. Esta activación es útil porque puede llegar a ser requerida para abrir una puerta cerrada del escenario. Por último existen nuevamente los mismos tipos de receptores de energía.
	
	\item Una puerta puede estar cerrada. Para abrirla se necesita cumplir con una cierta condición lógica que puede implicar el uso de los operadores lógicos \texttt{and}, \texttt{or} y \texttt{not} y se realiza sobre los botones y receptores de energía. Por ejemplo, puede haber una puerta que para ser abierta deba tener activado un receptor de energía y un botón. Para ello debería esperarse que una bola de energía sea emitida por un emisor hacia dicho receptor y colocar una roca u otro jugador sobre un botón.
	
	\item Se debe tener en cuenta que las bolas de energía, si bien tienen una dirección fija al ser lanzadas por el emisor, pueden toparse con un portal y teletransportarse además de cambiar su dirección al toparse con un bloque con dirección diagonal o rebotar sobre un bloque de metal. Sobre un bloque de roca no sucederá nada debido a que la asesina en el acto.

	\item Solamente se pueden colocar portales sobre bloques de metal. Intentar hacerlo sobre cualquier otro objeto es en vano.
	
	\item La roca se desintegra completamente al toparse con una barra de energía. Igualmente no representa en un peligro en sí para Chell.
	
	\item Pueden teletransportarse rocas por los portales de manera de acceder a otros lugares del escenario, donde puede ser necesario activar un botón. Para hacerlo, como se mencionó previamente se recomienda lanzar la roca hacia arriba o abandonarla si el portal está por debajo.
	
	\item Existe el objeto \texttt{pintool} para armonizar el trabajo en grupo y llegar al pastel. Este elemento se utiliza para que un jugador indique a otro que estaría bueno que cree un portal ahí para cumplir el objetivo del nivel. Queda a completa responsabilidad del usuario ser inteligente con el mismo y colocarlo de la manera más estratégica posible. Así mismo, mueren después de un determinado tiempo y no pueden crearse más de uno por jugador, por lo que la creación de un segundo pin tool en el escenario por parte del mismo jugador elimina inmediatamente el anterior.
	
	\item Se pueden resetear tanto los portales como los pintools de manera de utilizar otra combinación más estratégica.
\end{itemize}

\section{Apéndice de errores}




\end{document}